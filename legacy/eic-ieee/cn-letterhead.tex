\documentclass[11pt,english]{luclet}
\usepackage{times}
\usepackage[T1]{fontenc}
\usepackage[utf8]{inputenc}
\usepackage{hyperref}
\setlength\parskip{\medskipamount}
\setlength\parindent{0pt}

\makeatletter

\usepackage{fullpage}
\addtolength{\oddsidemargin}{0.25in}
\addtolength{\evensidemargin}{0.25in}
\addtolength{\textwidth}{-0.5in}

\date{\vspace*{0.8in}\today}

% My Information
\school{COLLEGE OF ARTS AND SCIENCES}
\dept{Department of Computer Science}
\office{George K. Thiruvathukal, Ph.D.\\Associate Professor}
\location{Water Tower Campus, LT-416B}
\address{820 North Michigan Avenue \\ Chicago, Illinois\ \ 60611, USA}
\telephone{+1 (312) 915-7986}
\fax{+1 (312) 915-7998}
\email{gkt@cs.luc.edu}
\http{http://www.cs.luc.edu/gkt}

\usepackage{babel}
\makeatother
\begin{document}

% Recipient Information

\letter{Ron Vetter\\
  Professor of Computer Science\\
  UNC Wilmington\\
  Wilmington, NC}

\signature{\includegraphics[scale=0.7]{signature.jpg}George K. Thiruvathukal}

\opening{Dear Ron:}

Please consider this letter as my application to be the next EIC of
Computing Now. Per the guidelines, I am keeping this statement brief.

As the next EIC, I would like to accomplish the following:

\begin{enumerate}

\item \textbf{Be able to accept original content:} Unlike the publications and
  transactions, that remain tied to a print publishing model (even
  with the steps taken to move them to digital), our focus will be on
  a more agile publishing workflow. We will pioneer the use of
  plaintext authoring workflows (e.g. based on approaches like
  Markdown and Restructured Text) to support end-to-end production of
  all things digital (web, PDF, e-Books) and print-on-demand
  technology. This workflow will allow volunteers and staff to produce
  original content at vastly lower cost, owing to the ability to do
  everything from submission to final production entirely with open
  source and freely available cloud-based solutions. 

\item \textbf{Be the pilot initiative for open access:} It is clear that the
  Computer Society absolutely needs to explore this. The train has
  already left the station, and many publishers and efforts that
  directly and indirectly compete with us are already doing open
  access. Computing Now will support open access by looking at the
  best emerging practices and look into using commercially-hosted or
  open-source stacks (e.g. DSpace) to support open access publishing,
  where authors of accepted papers pay a fee for publication based on
  the E-value (expectation) of digital sales.  This will also help CN
  to monetize its efforts. Wherever possible, we will partner with
  other emerging initiatives (e.g. new transactions, STCs, etc.) to be
  \emph{the} place where people will choose to partner. This will
  undoubtedly be an unprecedented opportunity for the staff, who will
  be able to learn and deploy technologies to support what is arguably
  the most exciting trend in publishing. (PLOS/1 is an enormously
  successful model, and in one scenario, the Computer Society may even
  want to co-brand its initiative with what PLOS/1 is doing. It is,
  after all, a platform that can host other work.)

\item \textbf{Revive CN Labs:} CN's early success was attributable to the
  enormous amount of volunteer energy that went into it. This energy
  still exists but, I'm afraid, has waned a bit, owing to the
  perception that many initiatives are trying to compete with CN as
  opposed to partner with CN. In addition, many of us have been called
  to serve in other capacities (STCs, various boards, etc.) which
  means that we need resources to scale our vision. As EIC, I'm going
  to work really hard to restore CN Labs, knowing all too well that
  innovation is what drives the long term value proposition of any
  group, no matter how big or small. It is clear to me that CN and CN
  Labs can do the strategic prototyping and concept proving unlike no
  other. We'll obviously demonstrate this through ambitious
  initiatives like accepting original content (through mechanisms
  other than Manuscript Central) and publishing content using a
  non-print workflow.  This will require us to spend effort evaluating
  various stacks, both in the open source and commercial spaces.

\item \textbf{Outsource IT for CN's needs only:} We will work with the IT group
  within the Computer Society to establish areas of common ground, but
  there is no doubt among our existing editorial board and this
  candidate that the present state of affairs is not working. CN,
  simply put, wants itself and other parts of the Computer Society to
  be viewed as a \emph{customer} of IT. We know the current IT director is
  extremely capable, but most of the energy seems to go toward
  maintaining servers and other functions that seemingly don't get us
  any closer to what we need for CN. CN needs technology to make it
  easier to do everything, from alternative member subscription models
  (below), to content management, to digital repositories, and
  integration with social technologies. For better or for worse, it
  simply isn't happening. As EIC, I will seek funding to explore
  externally hosted solutions for our needs, which can be viewed as a
  pilot investment for the rest of the Computer Society. We
  specifically should establish a strategy for establishing a completely
  separate web presence from the other Computer Society portals. I'd
  personally like to follow the model of the American Institute of
  Physics, which seemingly is doing things the ``right'' way when it
  comes to content hosting. I will speak in greater detail about this,
  should I be interviewed. If a non-computing society can do proper
  content hosting and domain/subdomain management, why can't we?

\item  \textbf{Improve member sign-up and benefits:} Lastly, CN cannot succeed
  unless there is a complete rethinking of membership. For new and
  current members, signup and renewal--even in a charitable
  analysis--is an exercise in tedium. If a student were to reach this
  page, they'd immediately turn away, because today's students have
  already grown accustomed to easy and quick sign-up through services
  like Facebook and Twitter. (In fact, the future of just about
  everything for us should be to sign in with your social ID to become
  a member.) In addition to sign-up, CN can play a major role to help
  shape the definition of member benefits, especially those that will
  be relevant to the next generation of computing professionals.
  Lastly, we will work to change the way that users are presented with
  the \$19 option to download an article. This link should be
  immediately changed to say something other than \$19 (e.g. see
  options). If someone does want to pay \$19, what should really
  happen is that the user is given a pro-rated short term
  membership. As I see it, a 2-3 month membership could be given for
  \$19. The IEEE and the Computer Society make money, and we
  potentially gain a new member once this period runs out. CN's motto
  is aimed at engagement, so we care deeply about these sort of issues
  and will work hard with all relevant groups to make the case that
  the present state of affairs is simply ``bad for business''.


\end{enumerate}

I wish to conclude by saying that I have way more ideas than I can
write about here. These are not the only ideas that you'll hear about,
and I will be happy to present a detailed plan as the next EIC in
consultation with my board. In many ways, CN is a perfect fit for me,
because it is one of the few places where my agile and off-my-feet
thinking can be put to good use. Nevertheless, I decided to focus on
these five, because we have unfinished business from the past few
years of CN that is not only important to use but is vital to the
future of the Computer Society in general. I'm excited about and
optimistic that these five initiatives can and will be implemented
under my watch. I hope that the Computer Society will be willing to
support me and our capable current and future editorial board to make
it happen.

\includegraphics[scale=0.7]{signature.jpg}\\
George K. Thiruvathukal, Ph.\ D.
\end{document}
