\documentclass[11pt,english]{luclet}
\usepackage{times}
\usepackage[T1]{fontenc}
\usepackage[utf8]{inputenc}
\usepackage{hyperref}
\setlength\parskip{\medskipamount}
\setlength\parindent{0pt}

\makeatletter

\usepackage{fullpage}
\addtolength{\oddsidemargin}{0.25in}
\addtolength{\evensidemargin}{0.25in}
\addtolength{\textwidth}{-0.5in}

\date{\vspace*{0.8in}\today}

% My Information
\school{COLLEGE OF ARTS AND SCIENCES}
\dept{Department of Computer Science}
\office{George K. Thiruvathukal, Ph.D.\\Associate Professor}
\location{Water Tower Campus, LT-416B}
\address{820 North Michigan Avenue \\ Chicago, Illinois\ \ 60611, USA}
\telephone{+1 (312) 915-7986}
\fax{+1 (312) 915-7998}
\email{gkt@cs.luc.edu}
\http{http://www.cs.luc.edu/gkt}

\usepackage{babel}
\makeatother
\begin{document}

% Recipient Information

\letter{Ron Vetter\\
  Professor of Computer Science\\
  UNC Wilmington\\
  Wilmington, NC}

\signature{\includegraphics[scale=0.7]{signature.jpg}George K. Thiruvathukal}

\opening{Dear Ron:}

This is my initial input to the process to select the next
editor in chief (EIC) of Computing Now, the online initiative of the Computer
Society in which I have served for 4 years as one of the associate EICs, where
I was appointed by the current EIC, Dejan Milojicic.

It is therefore with great pleasure and respect that I enter my
candidacy to serve as Computing Now's second EIC. Computing Now has
been able to accomplish a great deal in a short time and, I believe,
has established itself as a central point of *thought leadership*
within the Computer Society in general. On a personal note, it has
been greatly rewarding to me professionally. I do receive many e-mails
about the work we have done in Computing Now, nearly all of which are
positive and serve to motivate my continued involvement with this
important initiative.

Since joining the editorial board, I have been able to work on the following:

\begin{itemize}
  
\item one of the pioneers in blogging on a regular basis and co-developer
  of our social strategy with Christian Timmerer (with whom I co-chair
  an STC as well).

\item organizer of highly successful theme issues, including one of our
  most popular on GPGPU computing.

\item supervised a student project to develop CN/Android, an Android interface
  to CN content (developed by Matt Wojtowicz, graduate of Loyola University
  Chicago).

\item early interfacing with IT, especially when it came to Liferay hosting and usability matters.

\item social presence, co-developed with Christian Timmerer, also an AEIC.

\item pioneered the use of multimedia in a theme issue, which resulted in a
  substantial traffic being generated on the CN site.
  
\end{itemize}

Computing Now's original vision--a vision that remains important--is essentially to support our motto

\begin{quote}
  Access. Discover. Engage. 
\end{quote}

We've definitely gone a long way toward supporting the first two words
as a board. We still need to work on the third. I believe it goes
without saying that engagement is a twofold (and probably obvious)
challenge:

\begin{itemize}
  
\item getting our existing membership base excited about our services

\item convincing newer members, especially students and those who don't 
  pursue computer science and related engineering/technical degrees, that
  joining the Computer Society in general is "worth it".
\end{itemize}

I'm especially concerned about the second point, while acknowledging that 
we must do much more to address the first concurrently. What I observe today
as a university professor where there's a quality CS program is rather
distressing: Students simply aren't interested. I spoke about this phenomenon
at a workshop on Usability and User Engagement at Oxford University, where
I actually showed a slide from one of my students, who basically said that
he simply sees no value to the IEEE Computer Society membership from a 
student's point of view. This student, a top research student, mentioned
that in his view, most of the content from the CSDL can be gotten "by visiting
the actual researcher's home page" or "by visiting the university library". 

Somewhat paradoxically, I think it bears noting that university students
almost universally fail to comprehend that their library is actually *paying*
for the CSDL. Furthermore, they can continue to enjoy access (at least at my
university) by getting an alumni card, which gives them most of the valuable
privileges of being a Loyola University Chicago student--for life--at zero
cost.

The Computer Society, therefore, needs to realize that Computing Now is not
only a thought leader but might also be one of the few places where the
founding editorial board *truly gets it*. But we can't do it via the One-Eyed
Man theorem:

\begin{quote}
  In the land of the blind, the one eyed man is king.
\end{quote}

We must become a center of greater Computer Society investment and--dare I say
it--strategic risk taking. I believe I am one of the few candidates who will
emerge that can balance sensibility and risk taking to ensure that the CS
emerges stronger as a result of our efforts. Almost everyone who knows of my
work knows that I take my scholarship seriously and have a great sense of
tradition. At the same time, I have not allowed my thinking and skills to
deteriorate by working on the same old stuff all the time. In the past few
years, I have become a serious researcher in the emerging area of digital
humanities and have continued to be at the forefront of developments in my own
discipline. Our field is *supposed* to be about staying current and relevant,
and the Computer Society can and should play a significant role in this
regard, especially when it comes to practicing professionals.

It is herein that I think we see the second aspect of our disconnect with our
existing base. We absolutely need to reconnect with members on a different
level. Many of our member benefits, while valuable, are failing to reach a
higher level. Good examples include the profesional certifications and books.
While interesting, it is hardly the sort of benefit that one cannot get
anywhere else, and were I to hazard a guess, more people would find it easier
to subscribe to a digital collection like Safari on their own dime than
through the convoluted Computer Society membership signup and renewal process.

But here is where the disconnect becomes even more apparent: We're not social.
I was thinking about what I *wish* I could do today as a professor with my
membership. First, I wish I could share more easily. For example, whether I
have an institutional or individual digital library subscription, why can't I
do what I can do with my favorite publication, *The Economist*, and share an
article with a friend? That's right. I can't. And there are two immediate use
cases that come to mind:

\begin{itemize}
  
\item Student X contacts me from another institution (that doesn't have a CSDL
  subscription). I want to share my article with the student.

\item Colleague X contacts me from another institution. Often the colleague is in
  another discipline (science or humanities). I want to share my article with
  this colleague. 
\end{itemize}

In both cases, I cannot share easily without grabbing the PDF out of the CSDL
and sending it to these folks. Also in both cases, neither has the resources
or the inclination to be a member, although they might be willing to pay.

Then we get to the \$29 download problem. Because this is a touchy
subject for many within the Computer Society, I won't go into any kind
of detail. But I will say that I consider it a matter of utmost
importance for us to figure out how not to turn away prospective
members by showing them this \$29 price, which has to generate way more
than \$29 of bad will. (Even prospective members find this
distressing. At a minimum it should be raised with something like *see
options*.)

I could ramble on, but the above has been written not to whine but to
paint a picture of where some of the opportunities (and key
challenges/obstacles) lie for Computing Now. I believe that most of
our problems are of a social nature, or as I like to put it:

\begin{quote}
  Putting social back in the Computer Society
\end{quote}

So here are the things I would like to accomplish as the EIC of Computing Now:

\begin{enumerate}
  
\item Be able to accept original content through Manuscript Central

  Contrasted with the publications and transactions, however, we don't
  have a linear calendar. Our focus will be on accepting, reviewing, and
  publishing content in an ongoing fashion. We will have peer review to
  ensure the contributions will be valuable in tenure/promotion
  cases. Per the CRA guidelines, most CS departments are well on their
  way to getting digital artifacts to be accepted--even within
  colleges/schools that otherwise are stodgy when it comes to digital
  artifacts.

  It is my understanding that a Journal has already been given
  provisional approval to move to Phase 2 (based on my involvement as an
  at large member of the publications board). So as I see it, if such a
  journal is to be approved, it could be hosted and managed by CN and
  its editorial board.  We have a significant EB, and there is no reason
  that they cannot help with the workload of reviewing
  manuscripts. Because they are coming from all of the publications and
  elsewhere, there will almost always be someone who knows how to find
  the right reviewer for a topic.

\item Be the pilot initiative for open access

  Some would say we already had the model with DS Online. I'm going to
  respectfully disagree by saying it was too much, too early, and there
  wasn't enough technology to support it. Times have changed!

  I continue to worry that the Computer Society doesn't *get it* when it
  comes to open access publishing. It's not CSDL vs. Open Access. It's
  about CSDL *and/or* Open Access. The dream would be for it all to be
  Open Access, but this is not something that can or should happen
  overnight.

  We need to start somewhere. CN is that place. If CN can be the host to
  a journal with a flexible calendar, we can roll this out slowly and
  (hopefully) get it right in a way that is both good for CN and the CS.

  To not do it, however, puts the CS at enormous risk. There are an
  increasing number of hazards ahead, not the least of which could be
  some of the existing open access venues themselves,
  e.g. PLOS/1. Everyone is already painfully aware of this. For example,
  American Institute of Physics (larger than we are but not vastly
  larger) has already created AIP Advances. Recently accepted articles
  are already appearing there. They used a system called Polopoly to
  support this aspect of their operations without disrupting their other
  publications and journals. We need to draw inspiration from these
  models and embrace them.

\item Return to agility

  In my early days of CN, it was particularly exciting. We were doing so
  much out-of-box thinking and contemplating how we could help the
  Computer Society to "rule the universe". We were seemingly doing it
  all: developing mashups, mobile applications, blogging, special theme
  issues, and exploring other emerging technologies (Twitter, Facebook,
  etc.) And in most all cases, we enjoyed a fair amount of success. Then
  many of the ideas began to be duplicated elsewhere in the CS,
  sometimes making it impossible to distinguish from our own efforts.

  As a group, we all believe in what we are doing and are happy that the
  CS benefited from it. but there is a real question of what was really
  gained from a CN point of view. We and the CS are probably no richer
  for the methods we introduced, and we've kind of drifted from our
  agile focus by trying to be publication-like without the proper
  resources to support our vision fully.

  Consensus seems to have emerged in the board that CN is a lot like a
  Labs model for the Computer Society.  Much like the labs model still
  found at companies like Google, Avaya, HP, and IBM, CN should always
  be the goto place for input on emerging technologies. We can even be
  the place to prove concepts, especially through our students and
  developer communities.  So a real possibility is to bifurcate
  Computing Now to focus our efforts more coherently. To be agile again,
  we need a core of folks who work on CN Labs. To accept original
  content and enjoy the benefits of a publication, we need CN Digital
  Edition. I want to see us do both but think a subset of the AEICs and
  board needs to focus its energies on each initiative. Both CN Labs and
  CN Digital Edition are essential proofs of concept to shape the future
  of the Computer Society.

\item Outsource IT for CN's needs only

  It is abundantly clear that the relationship with IT needs work. The
  new director, Ray, is eminently capable and seems interested in
  working with us. At the same time, he and everyone needs to realize
  that IT needs to be not only about technology but must address other
  "soft" dimensions like usability and customer service. For better or
  for worse, I can say unequivocally that I don't feel that CN is a
  customer at this stage.

  My current feeling is that most of the services we need from IT really
  could be outsourced. I alluded earlier to Polopoly (without
  specifically recommending it) that is now being used by the American
  Institute of Physics. My university recently switched to TerminalFour
  for its content management, which seemingly has been written for
  scalability and performance (unlike Liferay) and uses XML for
  sophisticated templating and services interfaces. The Computer Society
  should think of itself as being more or less in an equivalence class
  with a university, given that the budgetary constraints and
  requirements have more overlap than difference.

  A focused IT, especially on the services side, would turn the
  organization into a team of integrators as opposed to developers. They
  could also be an important interface to commercial providers, much
  like what happens in most good university IT departments.

\item  Improve member sign-up and benefits

  I've already touched on membership. While not directly within our
  scope, we need it to be fixed. It is a major problem, especially from
  CN's point of view. If the students are not signing up, why not just
  give them a one-time membership for free (with some verification based
  on their student ID). Alternately, work with CS departments to give
  away some set number of memberships, say, to top graduates or
  whatever.

  Separately, as content is so vitally important to CN, it might make
  sense for us to incentivize student participation by giving them some
  credits toward membership based on their participation on CN or other
  CS sites.  This obviously requires an entirely different CMS
  technology than we have today, or it minimally requires that we
  leverage cross-site scripting where we have good metrics by
  user. Engagement will eventually happen over time, but it's unlikely
  to have with the existing membership base (who is not only growing
  older but often views social technologies with great suspicion if we
  are lucky or disdain otherwise). So it's only going to come from
  recent graduates, who tend to think something is *not* cool,
  especially if it doesn't work with social technologies they use
  (Facebook, Twitter, LinkedIn, Google Plus, among others).

  Lastly, I would interface with the membership committee to talk about
  this and the overall package of benefits, especially in terms of how
  it can be modernized so as to be attractive to the next generation.

\end{enumerate}

So those are just five areas where I think I can make a difference. I believe
I have the personality to *persuade* others, especially if they are willing to
me. Nothing I am talking about can come about without a much greater sense of
cooperation between volunteers and staff. Having been on the editorial boards
of CISE and CN and as an at large member of the Publications Board, I believe
that I have the skills to manage these relationships and help everyone to see
that all of us could do better if we are willing to open our minds and think
about the possibilities. 

In any event, I wish to conclude by thanking the committee for their time, the
present EIC, Dejan, for his extreme service to the Computer Society, and my
wonderful colleagues at Computing Now and Computing in Science and Engineering
(the publication). I have applied for both positions, knowing that I have a
strong vision for both, with the understanding that I would only be able to
accept one of them. I look forward to speaking with the committee to discuss
the future of Computing Now.

\includegraphics[scale=0.7]{signature.jpg}\\
George K. Thiruvathukal, Ph.\ D.
\end{document}
